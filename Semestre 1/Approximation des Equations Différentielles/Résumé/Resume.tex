\documentclass[10pt,a4paper]{report}
\usepackage[utf8]{inputenc}
\usepackage{amsmath}
\usepackage{amsfonts}
\usepackage{amssymb}
\usepackage{makeidx}
\usepackage{graphicx}
\usepackage{xcolor}
\usepackage[left=4cm,right=2cm,top=0cm,bottom=2cm]{geometry}
\newcommand*\acocher{\quitvmode{\fboxsep0pt \fboxrule0.8pt \fbox{\vrule height1.8ex depth.1ex width0pt \vrule height0pt depth0pt width1.9ex }}}


\begin{document}

\chapter*{Introduction}
\section*{Opérateur aux dérivées partielles}
Cas scalaire : opératoire aux dérivés partielles $P(x,D)$, symboles de l'opérateur $p(x,\xi)$, symbole principal $P_n(x,\xi)$, cône et vecteurs caractéristiques \textcolor{blue}{p1} \acocher \acocher ; Cas vectorielle : $P(x,D)$, elliptique, hyperbolique \textcolor{blue}{p1} \acocher \acocher ; 1er ordre et hyperbolisme \textcolor{red}{p2} \acocher \acocher ; symbole de Laplacien, de l'opérateur de la chaleur, de l'opérateur des ondes \textcolor{orange}{p2} \acocher \acocher ; elliptique, parabolique et diffusion, vitesse infini \textcolor{red}{p2} \acocher \acocher ; vitesse finie et infinie \textcolor{orange}{p2} \acocher \acocher ; hypersurface caractéristique \textcolor{blue}{p2} \acocher \acocher ; opérateur de type mixte Triconi \textcolor{red}{p2} \acocher \acocher ; 
\section*{Les principaux modèles étudiés}
Dirichlet \textcolor{red}{p3} \acocher \acocher ; Neumann \textcolor{red}{p3} \acocher \acocher ; Biharmonie (élasticité) \textcolor{red}{p3} \acocher \acocher ;  Stockes \textcolor{red}{p3} \acocher \acocher ; chaleur \textcolor{red}{p3} \acocher \acocher ; Transport \textcolor{red}{p4} \acocher \acocher ; équations des ondes \textcolor{red}{p4} \acocher \acocher \\
étapes pour faire une approximation d'EDP (5) \textcolor{orange}{p4} \acocher \acocher ; 
\chapter*{Méthodes des différences finies}
\section*{Principe de la méthode}
Différence finies en 1 D \textcolor{orange}{p1} \acocher \acocher ; exo $u^{(4)}()$ \textcolor{orange}{p1} \acocher \acocher ; dimension supérieure à 1 \textcolor{red}{p1} \acocher \acocher
\section*{Application à l'opérateur de Laplace complété}

\section*{Rappels et compléments matrices}
\section*{Convergence de la méthode des différences finies}
\chapter*{Méthodes des volumes finis}
\section*{Principe de la méthode}
\section*{La méthode volumes finis sur maillage cartésien}

\end{document}
