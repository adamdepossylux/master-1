\documentclass[10pt,a4paper]{report}
\usepackage[utf8]{inputenc}
\usepackage{amsmath}
\usepackage{amsfonts}
\usepackage{amssymb}
\usepackage{makeidx}
\usepackage{graphicx}
\usepackage{xcolor}
\usepackage[left=2cm,right=2cm,top=2cm,bottom=2cm]{geometry}
\newcommand*\acocher{\quitvmode{\fboxsep0pt \fboxrule0.8pt \fbox{\vrule height1.8ex depth.1ex width0pt \vrule height0pt depth0pt width1.9ex }}}


\begin{document}

\section*{Espace des probabilités}

Expérience aléatoire $\varepsilon$ \textcolor{blue}{p1} \acocher \acocher ; Univers $\Omega$ associé à $\varepsilon$ \textcolor{blue}{p1} \acocher \acocher ; univers dénombrable et finitude \textcolor{orange}{p1} \acocher \acocher \\
Évènement élémentaire, aléatoire \textcolor{blue}{p1} \acocher \acocher ; évènement dénombrable, infini \textcolor{orange}{p1} \acocher \acocher ; Évènement certain, impossible \textcolor{blue}{p2} \acocher \acocher ; $P(\Omega)$ \textcolor{blue}{p2} \acocher \acocher ; sous famille de partie de $\Omega$ \textcolor{red}{p2} \acocher \acocher ; tribut, $\sigma - algebre$ \textcolor{blue}{p2} \acocher \acocher ; espace probabilisable \textcolor{blue}{p2} \acocher \acocher ; tribu trivial, complet et le plus petit  \textcolor{blue}{p2} \acocher \acocher ; exemple de partition tribu \textcolor{orange}{p2} \acocher \acocher ; partition de $(A_1, A_2,..A_n)$ \textcolor{orange}{*p3} \acocher \acocher ; Espace de probabilité \textcolor{blue}{p3} \acocher \acocher ; $P(\emptyset)$, $P(A) + P(\overline{A})$, $\sum_{i=1,...,n}$ (demo \acocher \acocher) \textcolor{red}{p3} \acocher \acocher ; additivité et croissance de $P$ (demo \acocher \acocher) \textcolor{red}{p4} \acocher \acocher ; Inégalité de Boch \textcolor{red}{p4} \acocher \acocher ; croissance et décroissance des ensemble \textcolor{red}{p4} \acocher \acocher ; Formule de Poincarré \textcolor{red}{p4} \acocher \acocher ; probabilité à partir d'une famille de nombre ${w_i}$ (cas dénombrable \acocher \acocher) \textcolor{blue}{p5} \acocher \acocher ; probabilité uniforme \textcolor{blue}{p5} \acocher \acocher ; calcul de probabilité dans ce cas \textcolor{orange}{p5} \acocher \acocher  

\section*{Probabilité conditionnelle et indépendance}
Probabilité de $A$ sachant $B$ $P(A|B)$ \textcolor{red}{p1} \acocher \acocher ; nature de $P(A|B)$ \textcolor{red}{p1} \acocher \acocher ; formule des probabilités composées \textcolor{red}{p1} \acocher \acocher ; Théorème des probabilité totales \textcolor{red}{p1} \acocher \acocher ; Formule de Bayes (demo \acocher \acocher) \textcolor{red}{p2} \acocher \acocher ; 

\end{document}
