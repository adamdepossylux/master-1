\documentclass[10pt,a4paper]{report}
\usepackage[utf8]{inputenc}
\usepackage{amsmath}
\usepackage{amsfonts}
\usepackage{amssymb}
\usepackage{makeidx}
\usepackage{graphicx}
\usepackage{xcolor}
\usepackage[left=4cm,right=2cm,top=2cm,bottom=2cm]{geometry}
\newcommand*\acocher{\quitvmode{\fboxsep0pt \fboxrule0.8pt \fbox{\vrule height1.8ex depth.1ex width0pt \vrule height0pt depth0pt width1.9ex }}}


\begin{document}

\section*{Quelques exemples}
$\vartheta^\alpha$ \textcolor{blue}{p1} \acocher \acocher ; $\vartheta^\alpha f(x)$ \textcolor{blue}{p1} \acocher \acocher ; gradient \textcolor{blue}{p1} \acocher \acocher ; divergence $div(g)(x)$ \textcolor{blue}{p2} \acocher \acocher ; Equation aux dérivés partielles ordinaires et solutions \textcolor{blue}{p2} \acocher \acocher ; Problème de Cauchy \textcolor{blue}{p3} \acocher \acocher \\
L'équation de la chaleur \textcolor{blue}{p3} \acocher \acocher ; Schrödinger \textcolor{blue}{p3} \acocher \acocher ; Transport \textcolor{blue}{p3} \acocher \acocher ; Equation des ondes \textcolor{blue}{p3} \acocher \acocher ; Conditions de Dirichlet \textcolor{orange}{p4} \acocher \acocher ; dérivé normale \textcolor{blue}{p4} \acocher \acocher ; conditions de Neumann \textcolor{orange}{p4} \acocher \acocher \\
Elliptique (Laplace) \textcolor{blue}{p4} \acocher \acocher ; parabolique (équation de la chaleur) \textcolor{blue}{p4} \acocher \acocher ; hyperbolique (transport) \textcolor{blue}{p4} \acocher \acocher ;

\section*{Transformation de Fourier}


\end{document}
